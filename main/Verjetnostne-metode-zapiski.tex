\documentclass[a4paper, 12pt]{book}

\usepackage{fancyhdr}

\newcommand{\ttitle}{Verjetnostne metode v računalništvu - zapiski s predavanj prof. Marca}
\newcommand{\ttitleshort}{Verjetnostne metode v računalništvu}
\newcommand{\tauthor}{Tomaž Poljanšek}
\newcommand{\tdate}{študijsko leto 2023/24}

\usepackage{color}
\usepackage{soul}
\usepackage[numbers]{natbib}

\usepackage{physics}

\usepackage[parfill]{parskip}
\usepackage[hyphens]{url}

\usepackage[usestackEOL]{stackengine}[2013-10-15] % formatting Pascal
\usepackage[dvipsnames]{xcolor}

\usepackage{cancel}
\usepackage[export]{adjustbox}

% Related to math
\usepackage{amsmath,amssymb,amsfonts,amsthm}
\usepackage{mathtools}
\usepackage{youngtab}
\usepackage{tikz}

% encoding and language
\usepackage{lmodern}
\usepackage[slovene, english]{babel}
\usepackage[utf8]{inputenc}
\usepackage[T1]{fontenc}

% multiline comments
\usepackage{comment}
\usepackage{verbatim}

% random text - for texting
\usepackage{lipsum}
\usepackage{blindtext}

\usepackage{hyperref}

% images
\usepackage{graphicx}
\graphicspath{ {../images/} }

% no blank page
\usepackage{atbegshi}
\renewcommand{\cleardoublepage}{\clearpage}
%\renewcommand{\clearpage}{}

\usepackage{listings}
\usepackage{verbatim}
%\usepackage{fancyvrb}
%\usepackage{bera}

\newcommand*\Eval[3]{\left.#1\right\rvert_{#2}^{#3}}

\lstset{basicstyle=\ttfamily,
escapeinside={||},
mathescape=true}

% theorems
\theoremstyle{definition}
\newtheorem{counter}{Counter}[section]
\newtheorem{defn}[counter]{Definicija}
\newtheorem{lemma}[counter]{Lema}
\newtheorem{conseq}[counter]{Posledica}
\newtheorem{claim}[counter]{Trditev}
\newtheorem{theorem}[counter]{Izrek}
\newtheorem{pro}[counter]{Dokaz}
%%
\theoremstyle{remark}
\newtheorem*{ex}{Primer}
\newtheorem*{exmp}{Zgled}
\newtheorem*{rem}{Opomba}

% QED
\renewcommand\qedsymbol{$\blacksquare$}

\hypersetup{pdftitle={\ttitle}}

\addtolength{\marginparwidth}{-20pt}
\addtolength{\oddsidemargin}{40pt}
\addtolength{\evensidemargin}{-40pt}

\renewcommand{\baselinestretch}{1.3}
\setlength{\headheight}{15pt}
\renewcommand{\chaptermark}[1]
{\markboth{\MakeUppercase{\thechapter.\ #1}}{}} \renewcommand{\sectionmark}[1]
{\markright{\MakeUppercase{\thesection.\ #1}}} \renewcommand{\headrulewidth}{0.5pt} \renewcommand{\footrulewidth}{0pt}

% header
\fancyhf{}
\fancyhead[LE,RO]{\sl \thepage} 
\fancyhead[RE]{\sc \tauthor}
\fancyhead[LO]{\sc \ttitleshort}


\newcommand{\autfont}{\Large}
\newcommand{\titfont}{\LARGE\bf}
\newcommand{\clearemptydoublepage}{\newpage{\pagestyle{empty}\cleardoublepage}}
\setcounter{tocdepth}{1}

\newcommand{\N}{\mathbb{N}}
\newcommand{\Z}{\mathbb{Z}}
\newcommand{\Q}{\mathbb{Q}}
\newcommand{\R}{\mathbb{R}}
\newcommand{\C}{\mathbb{C}}
\newcommand{\ch}{\operatorname{char}}

\DeclarePairedDelimiter\ceil{\lceil}{\rceil}
\DeclarePairedDelimiter\floor{\lfloor}{\rfloor}

\usepackage{float}
\usepackage{multirow}
\usepackage{icomma}
\usepackage{tabularx}
\usepackage{hhline}

\usepackage{enumitem}
\usepackage{ulem}
\newcommand{\msout}[1]{\text{\sout{\ensuremath{#1}}}} % cross text in math mode

\usepackage{alltt}

\title{\ttitle}
\author{\tauthor}
\date{\tdate}

\newcommand\mymaketitle{
  \begin{titlepage}
    \begin{center}
        \vspace*{4cm}
        \Huge
        \textbf{\ttitle}
                        
        \vspace{1.5cm}
        \huge
        \tauthor
            
        \vspace{3cm}
        \Large
        \tdate
    \end{center}
  \end{titlepage}
}




\begin{document}

\selectlanguage{slovene}
%\setcounter{page}{1}
\renewcommand{\thepage}{}
\newcommand{\sn}[1]{"`#1"'}

\mymaketitle

\clearpage
%\AtBeginShipoutNext{\AtBeginShipoutDiscard}

\frontmatter

% kazalo
\pagestyle{empty}
\def\thepage{}
\tableofcontents{}

%%
\def\x{\hspace{3ex}}    %BETWEEN TWO 1-DIGIT NUMBERS
\def\y{\hspace{2.45ex}}  %BETWEEN 1 AND 2 DIGIT NUMBERS
\def\z{\hspace{1.9ex}}    %BETWEEN TWO 2-DIGIT NUMBERS
\stackMath

%\clearpage
%\phantomsection
%\addcontentsline{toc}{chapter}{Povzetek}
%\chapter*{Povzetek}

%Predloga.

%\newpage

\pagenumbering{arabic}

\mainmatter
\setcounter{page}{1}
\pagestyle{fancy}


% 1. predavanje: 6.10.


\chapter{Introduction}


\section{Probability}

$(\Omega, F, P_r)$:
\begin{itemize}[label=$\circ$]
  \item $\emptyset \in F$,
  \item $A \in F \implies A^c \in F$,
  \item $A_1, A_2 \dots \in F \implies \cup_{i=1}^{\infty} A_i \in F$.
\end{itemize}
$P_r(A) \geq 0$, \\
$P_r\left(\cup_{i=1}^{\infty} A_i\right) = \sum_{i=1}^{\infty} P_r(A_i)$ if $A_i$ disjoint, \\
$P_r\left(\cup_{i=1}^{\infty} A_i\right) \leq \sum_{i=1}^{\infty} P_r(A_i)$, \\
$\Omega = \{\omega_1, \omega_2 \dots\}$ - countable case. \\
$\begin{pmatrix}
  \omega_1 & \omega_2 & \dots \\
  p_1 & p_2 & \dots
\end{pmatrix}$
\begin{ex} \text{}
  \begin{verbatim}
    Alg():
      while True:
        B = sample as random from {0,1}  # 1 with probability p
        if B = 1:
          return
  \end{verbatim}
  $\Omega = \{1, 01, 001, 0001 \dots\}$ \\
  $\begin{pmatrix}
    1 & 01 & 001 & 0001 & \dots \\
    p & (1-p)p & (1-p)^2 p & (1-p)^3p & \dots
  \end{pmatrix}$.
\end{ex}


\section{Random variables}

$X: \Omega \to \Z$. \\
$E[X] = \sum_{c \in \Z} c \cdot P_r(X = c)$ expected value of $X$. \\
Properties:
\begin{itemize}[label=$\circ$]
  \item $E[f(X)] = \sum_{c \in \Z} f(c) \cdot P_r(X = c)$,
  \item $E[aX + bY] = aE[X] + bE[Y]$,
  \item $E[X \cdot Y] = E[X] \cdot E[Y]$ if $X, Y$ indepentent,
  \item $P_r(X \geq a) \leq \frac{E[X]}{a} \; \forall a > 0 \; X \geq 0$ Markov inequality.
\end{itemize}
\begin{ex}
  (Continuing from before). \\
  $X =$ number of trials before return. \\
  $X: \Omega \to \Z$. \\
  $X: 1 \to 1, 01 \to 2, 003 \to 3 \dots$ \\
  $\begin{pmatrix}
    1 & 2 & 3 & 4 & \dots \\
    p & (1-p)p & (1-p)^2 p & (1-p)^3p & \dots
  \end{pmatrix}$ - geometric distribution.
\end{ex}
\begin{claim}
  $E[X] = \frac{1}{p}$.
\end{claim}
\begin{pro}
  $X = \sum_{i=1}^{\infty} X_i$. \\
  $X_i = \begin{cases}
    1 \text{ if trial $i$ is executed} \\
    0 \text{ else}
  \end{cases}$ \\
  \begin{align*}
    E[X] &= E[\sum_{i=1}^{\infty} X_i] = \sum_{i=1}^{\infty} E[X_i] = \\
    &= \sum_{i=1}^{\infty} (1-p)^{i-1} = \frac{i=0}{\infty} (1-p)^i = \frac{1}{1-(1-p)} = \frac{1}{p}.
  \end{align*}
\end{pro}
$E[X] = \frac{1}{p}$. \\
$P_r(X \geq 100 \cdot \frac{1}{p}) \leq \frac{E[X]}{\frac{1}{p}} = \frac{1}{100}$.
\begin{defn}
  $H_n = 1 + \frac{1}{2} + \frac{1}{3} + \dots + \frac{1}{n} = \sum_{i=1}^{\infty} \frac{1}{i}$.
\end{defn}
\begin{theorem}
  $H_n \leq 1 + \ln(n)$.
\end{theorem}
\begin{pro}
  \begin{equation*}
    H_n = 1 + \sum_{i=2}^{n} \frac{1}{i} \stackrel{\text{integral}}{\leq}
    1 + \int_{1}^{n} \frac{dx}{x} = 1 + \Eval{\ln(x)}{1}{n} = 1 + \ln(n).
  \end{equation*}
  % skica
\end{pro}



\chapter{Quicksort, min-cut}


\section{Quicksort}

\begin{alltt}
  Input: set (no equal element) (unordered list) S\(\in\R\)
      (or whatever you can compare linearly)
  Output: ordered list
  Code:
    def Quicksort(S):
      if |S|= 0 or 1:
        return S
      else:
        a = uniformly at random from S
        S\(\sp{-}\) = \{b \(\in\) S | b < a\}
        S\(\sp{+}\) = \{b \(\in\) S | a < b\}
        return Quicksort(S\(\sp{-}\)), a, Quicksort(S\(\sp{+}\))
\end{alltt}
% skica
$C(n)$ - random variable, the number of comparisons in evaluation of Quicksort with $|S| = n$.
\begin{theorem}
  $E[C(n)] = O(N \log(n))$.
\end{theorem}
\begin{pro}
  $C(0) = C(1) = 0$. \\
  \begin{align*}
    E[C(n)] &= n - 1 + \sum_{i=1}^{n} \left(E[C(i-1)] + E[C(n-i)]\right) \cdot P_r(a \text{ is $i$-it element}) \leq \\
    &\leq n + \frac{2}{n} \sum_{i=1}^{n-1} E[C(i)].
  \end{align*}
  Induction: \\
  $n = 1: \checkmark$ \\
  $n-1 \to n$:
  \begin{align*}
    E[C(n)] &\leq n + \frac{2}{n} \sum_{i=1}^{n} E[C(i)] \leq \\
    &\leq n + \frac{2}{n} \sum_{i=1}^{n} 5i \log i \leq \\
    &\leq n + \frac{2}{n} \sum_{i=1}^{\lfloor\frac{n}{2}\rfloor} 5i \log i +
      \frac{2}{n} \sum_{i=1+\lfloor\frac{n}{2}\rfloor}^{n-1} 5i \log i \leq \\
    &\leq n + \frac{2}{n} \sum_{i=1}^{\lfloor\frac{n}{2}\rfloor} 5i \log \frac{n}{2} +
      \frac{2}{n} \sum_{i=1+\lfloor\frac{n}{2}\rfloor}^{n-1} 5i \log n \leq \\
    (\log \frac{n}{2} &= \log n - 1) \\
    &\leq n + \frac{2}{n} \left(\sum_{i=1}^n 5i \log n - \sum_{i=1}^{\frac{n}{2}} 5i\right) = \\
    &= n + \frac{10}{n} \left(\frac{n(n-1)}{2} \log n - \frac{\frac{n}{2} (\frac{n}{2} + 1)}{2}\right) \leq \\
    &\leq n + 5(n-1) \log n - n < \\
    &< 5n \log n.
  \end{align*}
\end{pro}
$P\left(C(n) \geq b \cdot 5n \log n\right) \stackrel{\text{Markov}}{\leq} \frac{1}{b}$.
\begin{pro} \text{} \\
  2: \\
  Let $S_1, S_2 \dots S_n$ sorted elements of $S$. \\
  Define random variable
  $X_{ij} = \begin{cases}
    1 \text{ if $S_i$ and $S_j$ are compared} \\
    0 \text{ else}
  \end{cases}$ \\
  $C(n) = \sum_{1 \leq i < j \leq n} E[X_{ij}]$. \\
  $E[X_{ij}] = P(S_i$ and $X_j$ compared$)$. \\
  % skica
  $S_{ij}$ - the last set including $S_i$ and $S_j$. \\
  $E[X_{ij}] = \frac{2}{|S_{ij}|} \leq \frac{2}{j-i+1}$. \\
  $|S_{ij}| \geq j - i + 1$. \\
  $S_{ij}$ has everything in between. \\
  \begin{align*}
    \implies E[C(n)] &\leq \sum_{1 \leq i < j \leq n} \frac{2}{j-i+1} = \\
    &\stackrel{k=j-i+1}{=} \sum_{i=1}^{n-1} \sum_{k=2}^{n-i+1} \frac{2}{k} \leq \\
    &\leq 2 \cdot n \cdot H_n \leq \\
    &\leq 2 n (1 + \log n).
  \end{align*}
\end{pro}



\clearpage
\phantomsection

\addcontentsline{toc}{chapter}{Literatura}
\bibliography{../bibtex/literatura}
\bibliographystyle{plainnat}


\clearpage
\phantomsection

%\chapter*{Dodatki}
%\addcontentsline{toc}{chapter}{Dodatki}
%D.




\end{document}
